\documentclass{article}

\title{How I scrape websites}
\author{Thomas Levine}

\newcommand\todo[1]{\textbf{#1}}

\begin{document}
\maketitle
\abstract{
I haven't yet found a guide to web scraping that explained most of
the things that I think should be explained. In particular, I have
a rather structured approach to writing scrapers, and I don't think
this approach is conveyed well in most guides to scraping. Thus, I
create this guide.
}
\section{Introduction}
\todo{Something general goes here.}

In section \ref{sec:structure}, I'll discuss how I structure my
scrapers in order to produce maintainable code.

In section \ref{sec:parsing}, I'll discuss how to navigate a variety
of source documents (mostly XML though) to extract structured
data. I'll also discuss how to use XPath and CSS selectors and how
to decide which one to use.
%Basically, use whichever one you know, use CSS if you're using
%jQuery and thus can't use XPath, and learn XPath if you don't
%know either.

In section \ref{sec:tools}, I'll discuss when how I decide which
languages, libraries, tools, \&c. to use for a given scraper.

In section \ref{sec:requests}, I'll discuss how to scrape websites
that use, cookies, POST requests or authentication. In handling
situations like these, I categorize my approaches into two categories:
(1) using a browser and (2) reverse-engineering website logic.

In section \ref{sec:database}, I'll discuss how to decide on a data
storage method and discuss schema designs for various data stores.

In section \ref{sec:errors}, I'll discuss various errors that come
up frequently and ways of handling such errors.

\section{Structure of a scraper}\label{sec:structure}
Dividing a program into small elements makes testing, collaboration
and iterative development easier. Some unnecessarily fancy names for
this are ``Don't repeat yourself (DRY)'', \todo{add more}.

For scrapers, natural divisions of program components occur at boundaries
between \emph{getting} data and \emph{parsing} data.
\footnote{You probably want to analyze/present the data too; that would be
another division, but I tend not to call that part ``scraping''.}

I'll go through a few scraper scenarios, from simple to complex, to
show how to I use the getting v. parsing criterion to divide scrapers
into manageable chunks.

\subsection{Scraping one file that is already downloaded}
Once you have the file on your computer, you can think of it as
just a plain text document (or some other type of document if it's
not plain text). All that's left to do is parse this text.
A proper walkthrough of this is in section \ref{sec:parsing},
but, essentially, you figure out the structure of the page in order
to find the data of interest and convert them to a simpler
representation that a computer can read more easily.

\subsection{Scraping one file that needs to be downloaded}
Let's say you don't have the file on your computer but you know
exactly how to get it.

In the simple case, you have a particular url that you need
to get it from, like maybe \url{http://thomaslevine.com}.
In this case, you can download the file in a line of code.

It gets more complicated if accessing the file involves a
more complex request. (This is discussed in section
\ref{sec:requests}.)

Either way, this is basically totally different from the parsing
and can easily be written as a separate component by a different person.

\subsection{Scraping many similar files that are already downloaded}\label{sec:many_downloaded}
If you're only scraping one page and you're not me, you'll probably just
copy and paste the information from that page rather than writing
a script to do it. Because of this, most scrapers are going to involve
multiple files.

To scrape multiple files, I add some logic to repeat my scraping code
across multiple files. One way of doing this is to wrap the parsing
code in a for-loop.

\subsubsection{Identifiers}
In order to do this, you need some sort of identifier for each page,
otherwise you would be overwriting something or you would not be able
to link a particular row to a particular page.

\paragraph{Implicit identifiers}
Let's say you that you only want aggregate data and thus don't need to link
a particular page to a particular row. For example, it might be that you have
thousands of music playlist files and that you are only wondering how long
each playlist is.

A dumb approach that does not involve explicitly setting an page identifier
is just to parse through the files in arbitrary order and append each of the
results to a database or file.

With this approach, it is harder to figure out where errors are occurring
and it is harder to restart a parse if it is stopped in the middle.

\paragraph{Explicit identifiers}
A simple way of getting a less arbitrary identifier is to save the file name,
the file path or the url from which the file was downloaded.

This approach isn't unique in all cases. Two different directories could
have files with the same name. Two different computers could have files with
the same path. And one url could present two different files depending on
the request that was made and the time at which the request was made.

If the file name or url is not unique, you can generate a unique number
for each file or page. This doesn't necessarily help with the tracking of
errors, but it does allow you to link data based on the page from which
they were scraped.

\paragraph{Representation of identifiers in a datastore}
\todo{
add a column\\
nosql\\
if you're using flat files
}

\subsection{Scraping many similar files that need to be downloaded}\label{sec:many_to_download}
As we saw above, parsing gets a bit more complicated when you have multiple files.
Getting the files also gets a bit complicated.

\subsubsection{Getting the files lazily or eagerly}
\todo{How do you decide whether to download them all at once first and save them or download them one-at-a-time and parse them one-at-a-time and maybe save them one-at-a-time?}

\subsubsection{Very predictable file urls}
If the pages are similar, they will probably have some similar url structure.
Sometimes, you can just increment a number to get to all of the pages.
For example, the website of the Cathedral at Santiago uses the very predictable
url structure of
\url{\verb+http://peregrinossantiago.es/eng/post-peregrinacion/estadisticas/?anio=%04d&mes=%02d+},
where \url{\%04d} is the year and \url{\%02d} is the month.
In such cases, you can list all of the urls pretty easily.

\subsubsection{Less predictable file urls}
File urls are not always that predictable. It might make sense to find a page
that lists urls for all of the files you want to download and then to download
each of the files.

For example, you might have a page that lists search results,
where each search result is a page that you want to parse.
In this case, I would get the search results page(s), parse each
of them to get the urls of the files of interest, then get and
parse each of the files.

\subsection{Scraping many files in a hierarchical page structure}
What if, instead of one list of search results, we had a store catalog
where items were grouped into departments? Such a situation is similar
to the search results except that there is a hierarchical structure.

The New York State statutes website is one such website. If I want to
get to the text of a particular statute, I go to the laws menu page,
then I click on the title, then I click on some section, then some
subsection, and so on.

In situations like this, I parse one webpage in order to get the information
needed to get the next webpage. For example, I scrape the list of articles
in order to get the url for each article.

\subsubsection{Why this is annoying}
One way of navigating this hierarchical structure would be to write a
separate page parser for each non-terminal node in the hierarchy.
For example, I would write a parser for the main laws menue, another
parser for the Arts and Cultural Affairs title, another for the Banking
title, \&c. For the New York State statutes, this would wind up being
over 9,000 scrapers.

\subsubsection{Why this isn't so bad}
Fortunately, such websites are normally automatically generated from
some database, so one adaptable scraper can normally handle the whole site.

Recall the structure in section \ref{sec:many_downloaded}
for parsing multiple similar pages. When you are scraping a hierarchy,
pages at the same level of the hierarchy are similar.
For example, two different articles (rather than an article and a sub-article)
on the New York statutes page have the same html structure,
so writing one function that can handle both of them is normally pretty easy.

Different levels of the hierarchy are likely to have rather similar
structures as well. There normally are some differences for the root
node and the end nodes, so you might run a slightly parsing function
on those nodes from the function you run on the other nodes.

\subsubsection{How this still is annoying}
Even if the different levels are similar, you will have to adapt your
function in order to handle the hierarchical information. This gets
a bit quite annoying. In particular,
\begin{itemize}
\item You normally want to keep track of the parents of a particular
node in the hierarchy, so database schema is more complicated.
\item Resuming a scrape on a crash can more complicated because you might
have several levels of loops.
\item Errors are more likely because you're trying to generalize a small
bit of code to many different pages.
\end{itemize}

\end{document}
